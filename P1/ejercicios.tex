\documentclass[11pt]{article}

\usepackage[utf8]{inputenc}
\usepackage[spanish, es-tabla]{babel}
\usepackage{mathtools}

%Gummi|065|=)
\title{\textbf{Práctica 1}}
\author{Pedro Bonilla Nadal\\
		Johanna Capote Robayna\\
		Guillermo Galindo Ortuño}
\date{}
\begin{document}

\maketitle

\textbf{Ejercicio 1}
Demuestra que la ecuaci\'on $x^3 + 4x^2 = 10 $ tiene una unica soluci\'on en el intervalo $ [1,2] $. Aproxima dicha ra\'iz con el m\'etodo de bisecci\'on con un error menor que $10^{-5}$. ¿Cu\'antas iteraciones ser\'an necesarias para conseguir un error menor que $10^{-8}$?
\\
\\
\textbf{Ejercicio 7}
Se considera la ecuaci\'on $x + \log x = 0$
\begin{itemize}

\item[a)]Prueba que dicha ecuaci\'on posee una \'unica soluci\'on.
\item[b)] Sea $a \in (0, \frac{1}{2})$. Prueba que  si $x_0 \in [a,1]$ en m\'etodo de Newton-Raphson es convergente.

\end{itemize}
\textbf{Ejercicio 16}
Demuestra que el sistema de ecuaciones
$$\begin{cases}
\frac{x^2y^2}{2} - x + \frac{7}{24} &= \ 0 \\
xy - y + \frac{1}{9} &= \ 0\\
\end{cases}$$
tiene una  ́unica soluci\'on en el intervalo $[0, 0.4]x[0, 0.4]$. Calcula una aproximaci\'on de la soluci\'on en el intervalo anterior mediante 4 iteraciones del m\'etodo de Newton partiendo de (0,0).
\end{document}
