\documentclass[11pt]{article}

\usepackage[utf8]{inputenc}
\usepackage[spanish, es-tabla, es-nodecimaldot]{babel}
\usepackage{mathtools}
\usepackage{amssymb}
\usepackage{amsthm}

\usepackage{dsfont}

%Gummi|065|=)
\title{\textbf{Práctica 1}}
\author{Pedro Bonilla Nadal\\
		Johanna Capote Robayna\\
		Guillermo Galindo Ortuño}
\date{}

%No indent
\setlength\parindent{0pt}
\begin{document}

\maketitle

\textbf{Ejercicio 1.}
Demuestra que la ecuaci\'on $x^3 + 4x^2 = 10 $ tiene una \'unica soluci\'on en el intervalo $ [1,2] $. Aproxima dicha ra\'iz con el m\'etodo de bisecci\'on con un error menor que $10^{-5}$. ¿Cu\'antas iteraciones ser\'an necesarias para conseguir un error menor que $10^{-8}$?
\\

\textbf{Solución.}

Para demostrar la existencia de esa solución nos basta con utilizar el teorema de Bolzano, ya que, como la funci\'on $f(x)=x^3 +4x^2 -10$ es continua de manera evidente, y $f(1) = -5$ y $f(2) =  14$, entonces existe un $c$ en el intervalo $(1,2)$ tal que $f(c) = 0$.

Ahora, para demostrar la unicidad estudiaremos la derivada de dicha función(es derivable en todo
 R). Tenemos que $f'(x) = 3x^2 +8x$, por tanto, es evidente que esa funcion no se anula en el intervalo $(1,2)$, luego $f(x)$ es estrictamente monótona en ese intervalo y solamente puede tener una raiz.
\\

Ahora, utilizando el programa para calcular la aproximación obtenida por el método de bisección que hemos diseñado tenemos que la raiz es: $1.36522674$.
\\

Supuesto que $x^*$ es la raiz buscada, que $x_n$ es la aproximación n-ésima, y que $\epsilon$ es el error máximo aceptable sabemos que:
$$|x_n - x^*| < \frac{b-a}{2^{n+1}}$$
Por tanto, despejando $n$ de
$$\frac{b-a}{2^{n+1}} = \epsilon$$
nos aseguramos que el error sea el deseado. Así, tras operar un poco, es suficiente con:
$$ n \geq log_2(\frac{b-a}{\epsilon})-1$$
En nuestro caso particular tenemos:
$$n \geq log_2(\frac{2-1}{10^{-8}})-1 \implies n \geq 26$$
\\ \\

\textbf{Ejercicio 7.}
Se considera la ecuaci\'on $x + \log x = 0$

\begin{itemize}
\item[a)]Prueba que dicha ecuaci\'on posee una \'unica soluci\'on.
\item[b)] Sea $a \in (0, \frac{1}{2})$. Prueba que  si $x_0 \in [a,1]$ en m\'etodo de Newton-Raphson es convergente.
\end{itemize}

\textbf{Soluci\'on.} \\
El dominio de nuestra funci\'on es: $\mathbb{R}^+$
\begin{itemize}
\item[\textbf{a)}]
Primero vemos que es \'unica:
$$f'(x) = 1 + \frac{1}{x} > 0 ,\ \forall x \in \mathbb{R}^+ $$
La funci\'on es creciente en todo su dominio, por lo tanto, en caso de existir una ra\'iz, esta es \'unica. \\ 
Ahora vamos a probar su existencia por Bolzano:

$$f(\frac{1}{e}) = \frac{1}{e} - 1 = -0,632120559< 0$$
$$f(e) = 1 + e > 0$$
Por el teorema de Bolzano sabemos que $\exists x_0 \in \mathbb{R} : f(x_0) = 0$

\item[\textbf{b)}]
Para probrar que $f(x) = x + \ln x$ es convergente tenemos que desmotrar que:
\begin{itemize}
\item[i)]$f(a) \cdot f(1) < 0$
\item[ii)]$\forall x \in [a,1], \ f'(x) \neq 0 $
\item[iii)]$f''(x) \text{ no cambia de signo en } [a,1] $
\end{itemize}

\begin{itemize}
\item[\textbf{i)}]
$$f(a) \cdot f(1) = f(a) \cdot 1 = f(a) $$

Como nuestra funci\'on es creciente, y $f(\frac{1}{2}) < 0$ se cumple que:
$$f(a) = a + \ln a < 0 , \  \forall a \in (0, \frac{1}{2}) $$
\item[\textbf{ii)}]
$$f'(x) = 1 + \frac{1}{x} $$
$$f'(x) \neq 0 , \forall x \in [a,1] $$
\item[\textbf{iii)}]
$$f''(x) = \frac{-1}{x^2} $$
$$f''(x) < 0 , \ \forall x \in [a,1] $$
\end{itemize}
\end{itemize}

\textbf{Ejercicio 16.}
Demuestra que el sistema de ecuaciones
$$\begin{cases}
\frac{x^2y^2}{2} - x + \frac{7}{24} &= \ 0 \\
xy - y + \frac{1}{9} &= \ 0\\
\end{cases}$$
tiene una unica soluci\'on en el intervalo $[0, 0.4]\times[0, 0.4]$. Calcula una aproximaci\'on de la soluci\'on en el intervalo anterior mediante 4 iteraciones del m\'etodo de Newton partiendo de (0,0).
\\

\textbf{Solución.}

Para demostrar que existe una única solución utilizaremos el teorema de
convergencia global. Para ello, transformaremos nuestro sistema en una ecuacion
de punto fijo, definida en un conjunto cerrado y acotado, que en nuestro caso
es $D = [0,0.4]\times[0,0.4]$

$$\begin{cases}
\frac{x^2y^2}{2} - x + \frac{7}{24} &= \ 0 \\
xy - y + \frac{1}{9} &= \ 0\\
\end{cases} \longrightarrow g(x,y) = (\frac{x^2y^2}{2} + \frac{7}{24}, xy +
\frac{1}{9} )$$

Una vez realiza esta transformación, necesitaremos varias condiciones para poder
aplicar el teorema.
\begin{itemize}
        \item $g(x) \in D ,\  \forall x \in D$\\
        Cada componente de $g$ es positiva, por lo que ambas son superiores que
        $0$ siempre. Ahora, es evidente que ambas alcanzan su máximo en el punto
        $(0.4, 0.4)$, con $g_1(0,4) = 0.304...$ y $g_2(0.4) = 0.27$, luego la condicion
        se cumple.
        \item g es contractiva\\
        Para demostrar que $g$ es contractiva utilizaremos un resultado de clase
        que afirma que si D es cerrado, acotado y convexo(en nuestro lo es) y

        $g:D \rightarrow \mathds{R}$ es de clase 1 y existe un L tal que
        $||Jg(x)|| \leq L < 1$ para todo $x$ en D, entonces g es contractiva.\\

        Podriamos utilizar cualquier norma, pero por comodidad utilizaremos la
        norma del máximo.
        $$Jg(x) =\begin{pmatrix}
 y^2x & x^2y    \\
 y    & x       \\
\end{pmatrix}        $$

Y en nuestro intervalo $||Jg(x)|| \leq 0.8 <1$ por lo que es contractiva y por lo tanto tiene solución única.
\end{itemize}
\end{document}
