\documentclass[11pt,a4paper]{article}

% Packages
\usepackage[utf8]{inputenc}
\usepackage[spanish, es-tabla]{babel}
\usepackage{caption}
\usepackage{listings}
\usepackage{adjustbox}
\usepackage{enumitem}
\usepackage{boldline}
\usepackage{amssymb, amsmath}
\usepackage[margin=1in]{geometry}
\usepackage{xcolor}
\usepackage{soul}

\usepackage{listings}
\usepackage{color}

\definecolor{mygreen}{rgb}{0,0.6,0}
\definecolor{mygray}{rgb}{0.5,0.5,0.5}
\definecolor{mymauve}{rgb}{0.58,0,0.82}

\lstset{
  backgroundcolor=\color{white},   % choose the background color; you must add \usepackage{color} or \usepackage{xcolor}; should come as last argument
  basicstyle=\footnotesize,        % the size of the fonts that are used for the code
  breakatwhitespace=false,         % sets if automatic breaks should only happen at whitespace
  breaklines=true,                 % sets automatic line breaking
  captionpos=b,                    % sets the caption-position to bottom
  commentstyle=\color{mygreen},    % comment style
  deletekeywords={...},            % if you want to delete keywords from the given language
  escapeinside={\%*}{*)},          % if you want to add LaTeX within your code
  extendedchars=true,              % lets you use non-ASCII characters; for 8-bits encodings only, does not work with UTF-8
  frame=single,	                   % adds a frame around the code
  keepspaces=true,                 % keeps spaces in text, useful for keeping indentation of code (possibly needs columns=flexible)
  keywordstyle=\color{blue},       % keyword style
  language=Octave,                 % the language of the code
  morekeywords={*,...},            % if you want to add more keywords to the set
  numbers=left,                    % where to put the line-numbers; possible values are (none, left, right)
  numbersep=5pt,                   % how far the line-numbers are from the code
  numberstyle=\tiny\color{mygray}, % the style that is used for the line-numbers
  rulecolor=\color{black},         % if not set, the frame-color may be changed on line-breaks within not-black text (e.g. comments (green here))
  showspaces=false,                % show spaces everywhere adding particular underscores; it overrides 'showstringspaces'
  showstringspaces=false,          % underline spaces within strings only
  showtabs=false,                  % show tabs within strings adding particular underscores
  stepnumber=2,                    % the step between two line-numbers. If it's 1, each line will be numbered
  stringstyle=\color{mymauve},     % string literal style
  tabsize=2,	                   % sets default tabsize to 2 spaces
  title=\lstname                   % show the filename of files included with \lstinputlisting; also try caption instead of title
}

% Meta
\title{Ejemplos y código del método de bisección.}
\author{Pedro Bonilla Nadal\\
  Johanna Capote Robayna\\
  Guillermo Galindo Ortuño}
\date{\today}

% Custom
\providecommand{\abs}[1]{\lvert#1\rvert}
\setlength\parindent{0pt}
\definecolor{Light}{gray}{.90}
\newcommand\ddfrac[2]{\frac{\displaystyle #1}{\displaystyle #2}}

\begin{document}

\maketitle

\section{Ejemplos.} Busca tres ecuaciones no lineales diferentes $f (x) = 0$, verificando $ |f'(x^* )| \simeq 1$ , $ |f'(x^*)| >> 1$ , $ |f'(x^* )| << 1,$ respectivamente, de las que puedas obtener una solución exacta y un intervalo para cada una de ellas que contenga una única solución. Aproxima la misma en cada caso con el método de bisección fijando un error máximo y analiza la diferencia entre el número de iteraciones realizadas con cada una de las variantes \emph{a), b), c)} y \emph{d) }del método.\\

\emph{Solución.}
\begin{itemize}
\item $f_1(x) = e^x-1 \implies f_1(0) = 0,\ f_1'(0)=1\ \simeq 1$
\begin{verbatim}
 Introduce el primer valor del intervalo: -1
 Introduce el segundo valor del intervalo: 2
 Introduzca la funcion de la que quiere hallar una raiz en
 el intervalo anterior:
 f(x) := np.exp(x)-1

 Criterio 1:
 0.5
 -0.25
 0.125
 -0.0625
 0.03125
 -0.015625
 0.0078125
 -0.00390625
 0.001953125
 -0.0009765625
 0.00048828125
 -0.000244140625
 0.0001220703125
 -6.103515625e-05
 3.0517578125e-05
 -1.52587890625e-05
 7.62939453125e-06
 -3.814697265625e-06
 1.9073486328125e-06
 1.9073486328125e-06

 Criterio 2:
 0.5
 -0.25
 0.125
 -0.0625
 0.03125
 -0.015625
 0.0078125
 -0.00390625
 0.001953125
 -0.0009765625
 0.00048828125
 -0.000244140625
 0.0001220703125
 -6.103515625e-05
 3.0517578125e-05
 -1.52587890625e-05
 7.62939453125e-06
 7.62939453125e-06

 Criterio 3:
 0.5
 -0.25
 0.125
 -0.0625
 0.03125
 -0.015625
 0.0078125
 -0.00390625
 0.001953125
 -0.0009765625
 0.00048828125
 -0.000244140625
 0.0001220703125
 -6.103515625e-05
 3.0517578125e-05
 -1.52587890625e-05
 7.62939453125e-06
 7.62939453125e-06

 Criterio 4:
 17
 0.5
 -0.25
 0.125
 -0.0625
 0.03125
 -0.015625
 0.0078125
 -0.00390625
 0.001953125
 -0.0009765625
 0.00048828125
 -0.000244140625
 0.0001220703125
 -6.103515625e-05
 3.0517578125e-05
 -1.52587890625e-05
 7.62939453125e-06
 7.62939453125e-06

  \end{verbatim}

\item $f_2(x) = e^x-e^15 \implies f_2(10) = 0,\ f_2'(15)=e^15\ >> 1$
\begin{verbatim}
Introduce el primer valor del intervalo: 0
Introduce el segundo valor del intervalo: 3
Introduzca la funcion de la que quiere hallar una raiz en el intervalo anterior
f(x) := np.exp(x) - np.exp(1)
Criterio 1:
1.5
0.75
1.125
0.9375
1.03125
0.984375
1.0078125
0.99609375
1.001953125
0.9990234375
1.00048828125
0.999755859375
1.0001220703125
0.99993896484375
1.000030517578125
0.9999847412109375
1.0000076293945312
0.9999961853027344
1.0000019073486328
1.0000019073486328
Criterio 2:
1.5
0.75
1.125
0.9375
1.03125
0.984375
1.0078125
0.99609375
0.99609375
1.001953125
0.9990234375
1.00048828125
0.999755859375
1.0001220703125
0.99993896484375
1.000030517578125
0.9999847412109375
1.0000076293945312
0.9999961853027344
1.0000019073486328
1.0000019073486328
Introduce una raiz: 1
Criterio 3:
1.5
0.75
1.125
0.9375
1.03125
0.984375
1.0078125
0.99609375
1.001953125
0.9990234375
1.00048828125
0.999755859375
1.0001220703125
0.99993896484375
1.000030517578125
0.9999847412109375
1.0000076293945312
1.0000076293945312
Criterio 4:
17
1.5
0.75
1.125
0.9375
1.03125
0.984375
1.0078125
0.99609375
1.001953125
0.9990234375
1.00048828125
0.999755859375
1.0001220703125
0.99993896484375
1.000030517578125
0.9999847412109375
1.0000076293945312
1.0000076293945312
39 - 15.00000000005457
40 - 15.000000000009095
41 - 14.999999999986358
42 - 14.999999999997726
14.999999999997726
Introduce una raiz: 15
Criterio 3:
1 - 25.0
2 - 12.5
3 - 18.75
4 - 15.625
5 - 14.0625
6 - 14.84375
7 - 15.234375
8 - 15.0390625
9 - 14.94140625
10 - 14.990234375
11 - 15.0146484375
12 - 15.00244140625
13 - 14.996337890625
14 - 14.9993896484375
15 - 15.00091552734375
16 - 15.000152587890625
17 - 14.999771118164062
18 - 14.999961853027344
19 - 15.000057220458984
20 - 15.000009536743164
15.000009536743164
Criterio 4:
21
1 - 25.0
2 - 12.5
3 - 18.75
4 - 15.625
5 - 14.0625
6 - 14.84375
7 - 15.234375
8 - 15.0390625
9 - 14.94140625
10 - 14.990234375
11 - 15.0146484375
12 - 15.00244140625
13 - 14.996337890625
14 - 14.9993896484375
15 - 15.00091552734375
16 - 15.000152587890625
17 - 14.999771118164062
18 - 14.999961853027344
19 - 15.000057220458984
20 - 15.000009536743164
21 - 14.999985694885254
14.999985694885254

\end{verbatim}
\item $f_3(x) = ln(x)^{19}-1 \implies f_3(e) = 0,\ f_3'(e)=19\frac{1}{e^18}\ << 1$
\begin{verbatim}
Introduce el primer valor del intervalo: 0
Introduce el segundo valor del intervalo: 50
Introduzca la funcion de la que quiere hallar una raiz en el intervalo anterior
f(x) := -np.exp(x) + np.exp(15)
Criterio 1:
1 - 25.0
2 - 12.5
3 - 18.75
4 - 15.625
5 - 14.0625
6 - 14.84375
7 - 15.234375
8 - 15.0390625
9 - 14.94140625
10 - 14.990234375
11 - 15.0146484375
12 - 15.00244140625
13 - 14.996337890625
14 - 14.9993896484375
15 - 15.00091552734375
16 - 15.000152587890625
17 - 14.999771118164062
18 - 14.999961853027344
19 - 15.000057220458984
20 - 15.000009536743164
21 - 14.999985694885254
22 - 14.999997615814209
23 - 15.000003576278687
15.000003576278687
Criterio 2:
1 - 25.0
2 - 12.5
3 - 18.75
4 - 15.625
5 - 14.0625
6 - 14.84375
7 - 15.234375
8 - 15.0390625
9 - 14.94140625
10 - 14.990234375
11 - 15.0146484375
12 - 15.00244140625
13 - 14.996337890625
14 - 14.9993896484375
15 - 15.00091552734375
16 - 15.000152587890625
17 - 14.999771118164062
18 - 14.999961853027344
19 - 15.000057220458984
20 - 15.000009536743164
21 - 14.999985694885254
22 - 14.999997615814209
23 - 15.000003576278687
24 - 15.000000596046448
25 - 14.999999105930328
26 - 14.999999850988388
27 - 15.000000223517418
28 - 15.000000037252903
29 - 14.999999944120646
30 - 14.999999990686774
31 - 15.000000013969839
32 - 15.000000002328306
33 - 14.99999999650754
34 - 14.999999999417923
35 - 15.000000000873115
36 - 15.00000000014552
37 - 14.999999999781721
38 - 14.99999999996362
39 - 15.00000000005457
40 - 15.000000000009095
41 - 14.999999999986358
42 - 14.999999999997726
14.999999999997726
Introduce una raiz: 15
Criterio 3:
1 - 25.0
2 - 12.5
3 - 18.75
4 - 15.625
5 - 14.0625
6 - 14.84375
7 - 15.234375
8 - 15.0390625
9 - 14.94140625
10 - 14.990234375
11 - 15.0146484375
12 - 15.00244140625
13 - 14.996337890625
14 - 14.9993896484375
15 - 15.00091552734375
16 - 15.000152587890625
17 - 14.999771118164062
18 - 14.999961853027344
19 - 15.000057220458984
20 - 15.000009536743164
15.000009536743164
Criterio 4:
21
1 - 25.0
2 - 12.5
3 - 18.75
4 - 15.625
5 - 14.0625
6 - 14.84375
7 - 15.234375
8 - 15.0390625
9 - 14.94140625
10 - 14.990234375
11 - 15.0146484375
12 - 15.00244140625
13 - 14.996337890625
14 - 14.9993896484375
15 - 15.00091552734375
16 - 15.000152587890625
17 - 14.999771118164062
18 - 14.999961853027344
19 - 15.000057220458984
20 - 15.000009536743164
21 - 14.999985694885254
14.999985694885254

\end{verbatim}
\end{itemize}

\newpage

\section{Anexo-Código.}

\lstinputlisting[language=Python]{../source/Ejercicio1.py}

\end{document}
