\documentclass[11pt,a4paper]{article}

% Packages
\usepackage[utf8]{inputenc}
\usepackage[spanish, es-tabla]{babel}
\usepackage{caption}
\usepackage{listings}
\usepackage{adjustbox}
\usepackage{enumitem}
\usepackage{boldline}
\usepackage{amssymb, amsmath}
\usepackage[margin=1in]{geometry}
\usepackage{xcolor}
\usepackage{soul}

% Meta
\title{Título}
\author{Pedro Bonilla Nadal}
\date{\today}

% Custom
\providecommand{\abs}[1]{\lvert#1\rvert}
\setlength\parindent{0pt}
\definecolor{Light}{gray}{.90}
\newcommand\ddfrac[2]{\frac{\displaystyle #1}{\displaystyle #2}}

\begin{document}
\maketitle
	2. \textbf{Ejemplos.} Busca tres ecuaciones no lineales diferentes $f (x) = 0$, verificando $ |f'(x^* )| \simeq 1$ , $ |f'(x^*)| >> 1$ , $ |f'(x^* )| << 1,$ respectivamente, de las que puedas obtener una solución exacta y un intervalo para cada una de ellas que contenga una única solución. Aproxima la misma en cada caso con el método de bisección fijando un error máximo y analiza la diferencia entre el número de iteraciones realizadas con cada una de las variantes \emph{a), b), c)} y \emph{d) }del método.\\

\emph{Solución.}
\begin{itemize}
	\item $f_1(x) = e^x-1 \implies f_1(0) = 0,\ f_1'(0)=1\ \simeq 1$
	
	
	
	\item $f_2(x) = e^x-e \implies f_2(1) = 0,\ f_2'(1)=e\ >> 1$
	\item $f_3(x) = ln(x)-1 \implies f_3(e) = 0,\ f_3'(e)=\frac{1}{e}\ << 1$
\end{itemize}

	

\end{document}
