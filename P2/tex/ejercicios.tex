\documentclass[11pt]{article}

\usepackage[utf8]{inputenc}
\usepackage[spanish, es-tabla, es-nodecimaldot]{babel}
\usepackage{mathtools}
\usepackage{amssymb}
\usepackage{amsthm}

\usepackage{dsfont}

%Gummi|065|=)
\title{\textbf{Práctica 2}}
\author{Pedro Bonilla Nadal\\
		Johanna Capote Robayna\\
		Guillermo Galindo Ortuño}
\date{}

%No indent
\setlength\parindent{0pt}
\begin{document}

\maketitle

\textbf{Ejercicio 5.} Calcula el orden de precisión con respecto a $h$ de las siguiente fórmulas para la aproximación numérica de $f'(c)$: \\ \\
\textbf{a)} $f'(c) \simeq \frac{1}{6h}(-11f(c) + 18f(c + h) -9f(c + 2h) +2f(c+3h))$ \\ \\
\textbf{solución} \\
\underline{orden 1:} \\
$$f(c + h) = f(c) + hf'(c) \ ; \ f(c + 2h) = f(c) + 2hf'(c) \ ; \ f'(c + 3h) = f(c) + 3hf'(c)$$
$$f'(c) = \frac{1}{6h}(-11f(c) + 18  [f(c) + hf'(c)] - 9  [f(c) + 2hf'(c)] + 2 [f(c) + 3hf'(c)]) = $$
$$= \frac{1}{6h}(-11f(c) + 18f(c) - 9f(c) + 2f(c) + 18hf'(c) - 18hf'(c) + 6hf'(c)) = $$
$$= \frac{1}{6h}(6hf'(c)) = f'(c) $$
\underline{orden 2:} \\
$$ f(c + h) = f(c) + hf'(c) + \frac{h^{2}}{2!} f''(c)$$ $$ f(c + 2h) = f(c) + 2hf'(c) + \frac{4h^{2}}{2!} f''(c)$$ $$f'(c + 3h) = f(c) + 3hf'(c) + \frac{9h^{2}}{2!} f''(c) $$
$f'(c) = \frac{1}{6h}(-11f(c) + 18  [f(c) + hf'(c) + \frac{h^{2}}{2!} f''(c)] - 9  [f(c) + 2hf'(c) + \frac{4h^{2}}{2!} f''(c)] + 2 [f(c) + 3hf'(c) + \frac{9h^{2}}{2!} f''(c)]) = \frac{1}{6h}(-11f(c) + 18f(c) - 9f(c) + 2f(c)  + 18hf'(c) - 18hf'(c) + 6hf'(c) + f''(c)[\frac{18h^{2}}{4} - 9h^{2} + \frac{18h^{2}}{4}]) = \frac{1}{6h}(6hf'(c) + f''(c)(\frac{18h^{2}}{2} + 9h^{2})) = f'(c)$ \\
\underline{orden 3:} \\
$$ f(c + h) = f(c) + hf'(c) + \frac{h^{2}}{2!} f''(c) + \frac{h^{3}}{3!}f'''(c)$$
$$ f(c + 2h) = f(c) + 2hf'(c) + \frac{4h^{2}}{2!} f''(c) + \frac{8h^{3}}{3!}f'''(c)$$ 
$$f'(c + 3h) = f(c) + 3hf'(c) + \frac{9h^{2}}{2!} f''(c) + \frac{27h^{3}}{3!}f'''(c)$$
$f'(c) = \frac{1}{6h}(6hf'(c) + f'''(c)[\frac{18h^{3}}{3!} - \frac{72h^{3}}{3!} + \frac{54h^{3}}{3!}]) =  \frac{1}{6h}(6hf'(c) + f'''(c)[3h^{3} - 12h^{3} + 9h^{3}]) = f'(c)  $ \\
\underline{orden 4:} \\
$$ f(c + h) = f(c) + hf'(c) + \frac{h^{2}}{2!} f''(c) + \frac{h^{3}}{3!}f'''(c) + \frac{h^{4}}{4!}f^{v}(c)$$
$$ f(c + 2h) = f(c) + 2hf'(c) + \frac{4h^{2}}{2!} f''(c) + \frac{8h^{3}}{3!}f'''(c) + \frac{16h^{4}}{4!}f^{v}(c)$$ 
$$f'(c + 3h) = f(c) + 3hf'(c) + \frac{9h^{2}}{2!} f''(c) + \frac{27h^{3}}{3!}f'''(c) + \frac{81h^{4}}{4!}f^{v}(c)$$
$f'(c) = \frac{1}{6h}(6hf'(c) + f^{v}(c)[\frac{18h^{4}}{4!} - \frac{144h^{4}}{4!} + \frac{162h^{4}}{4!}]) = \frac{1}{6h}(6hf'(c) + f^{v}(c)\frac{36h^{4}}{24}) \neq f'(c) $ 
\end{document}
