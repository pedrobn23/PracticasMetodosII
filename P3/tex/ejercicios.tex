\documentclass[11pt]{article}

\usepackage[utf8]{inputenc}
\usepackage[spanish, es-tabla, es-nodecimaldot]{babel}
\usepackage{mathtools}
\usepackage{amssymb}
\usepackage{amsthm}

\usepackage{dsfont}

%Gummi|065|=)
\title{\textbf{Práctica 3}}
\author{Pedro Bonilla Nadal\\
		Johanna Capote Robayna\\
		Guillermo Galindo Ortuño}
\date{}

%No indent
\setlength\parindent{0pt}
\begin{document}

\maketitle

\textbf{Ejercicio 4.} Analizamos en este ejercicio el método de Crank-Nicholson: \\

\textbf{c)} Calcula la solución exacta del problema y analiza los errores cometidos con la aproximación comparándolos a los obtenidos por el método de Euler mejorado.

$$P.V.I  \rightarrow  \left\lbrace
y'(t) = y - t^2 \  ;\  t \in [0,2] \atop
y(0) = 3
\right. $$

\textbf{Solución.}
$$y(t) = t^2 + 2 t + e^t + 2$$

\begin{table}[h]
\centering

\begin{tabular}{|l|l|l|l|}
\hline
Crank-Nicholson           & Error                 & Euler Mejorado     & Error           \\ \hline
3                  & 0.0                   & 3                  & 0.0                    \\ \hline
3.6622222222222223 & 0.0008194640620522442 & 3.658              & 0.003402758160170105   \\ \hline
4.4538271604938275 & 0.0020024628525572297 & 4.44396            & 0.007864697641270624   \\ \hline
5.385788751714678  & 0.0036699513241682524 & 5.3684312          & 0.013687600390509758   \\ \hline
6.471519585429051  & 0.005978656936583171  & 6.444286064        & 0.021254864492467718   \\ \hline
7.727412826635507  & 0.009130998176462235  & 7.68722899808      & 0.031052830379044494   \\ \hline
9.173504565887841  & 0.013387643151292039  & 9.1164193776576    & 0.0436975457894896     \\ \hline
10.834283358307362 & 0.019083391462684673  & 10.755231640742272 & 0.05996832610024051    \\ \hline
12.739679660153442 & 0.026647235758325394  & 12.6321826017      & 0.08084982268954377    \\ \hline
14.92627514018754  & 0.03662767577459469   & 14.7820627740808   & 0.10758469033214624    \\ \hline
17.43878072689588  & 0.04972462796522947   & 17.247316584378574 & 0.14173951455207856    \\ \hline
\end{tabular}
\end{table}
\end{document}
