\documentclass[11pt]{article}

\usepackage[utf8]{inputenc}
\usepackage[spanish, es-tabla, es-nodecimaldot]{babel}
\usepackage{mathtools}
\usepackage{amssymb}
\usepackage{amsthm}

\usepackage{dsfont}

%Gummi|065|=)
\title{\textbf{Práctica 3}}
\author{Pedro Bonilla Nadal\\
		Johanna Capote Robayna\\
		Guillermo Galindo Ortuño}
\date{}

%No indent
\setlength\parindent{0pt}
\begin{document}

\maketitle

\textbf{Ejercicio 4.} Analizamos en este ejercicio el método de Crank-Nicholson: \\

\textbf{c)} Calcula la solución exacta del problema y analiza los errores cometidos con la aproximación comparándolos a los obtenidos por el método de Euler mejorado.

$$P.V.I  \rightarrow  \left\lbrace
y'(t) = y - t^2 \  ;\  t \in [0,2] \atop
y(0) = 3
\right. $$

\textbf{Solución.}
$$y(t) = t^2 + 2 t + e^t + 2$$

\begin{table}[h]
\centering

\begin{tabular}{|l|l|l|l|}
\hline
Grupo 2            & Error                 & Euler Mejorado     & Error               \\ \hline
3                  & 0.0                   & 3                  & 0.0                 \\ \hline
3.6622222222222223 & 0.0008194640620522442 & 3.718              & 0.05659724183982995 \\ \hline
4.4538271604938275 & 0.0020024628525572297 & 4.59072            & 0.13889530235872982 \\ \hline
5.385788751714678  & 0.0036699513241682524 & 5.6360928          & 0.2539739996094905  \\ \hline
6.471519585429051  & 0.005978656936583171  & 6.876355072        & 0.4108141435075323  \\ \hline
7.727412826635507  & 0.009130998176462235  & 8.33908028928      & 0.6207984608209554  \\ \hline
9.173504565887841  & 0.013387643151292039  & 10.0584595587072   & 0.8983426359706517  \\ \hline
10.834283358307362 & 0.019083391462684673  & 12.076889852796928 & 1.2616898859522507  \\ \hline
12.739679660153442 & 0.026647235758325394  & 14.44694341746819  & 1.7339109930730743  \\ \hline
14.92627514018754  & 0.03662767577459469   & 17.23380983766056  & 2.344162373247613   \\ \hline
17.43878072689588  & 0.04972462796522947   & 20.518324198699094 & 3.129268099768442   \\ \hline
\end{tabular}
\end{table}
\end{document}
